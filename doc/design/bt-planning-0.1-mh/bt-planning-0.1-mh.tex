\documentclass[a4paper, 10pt]{article}

\usepackage[dutch]{babel}
\usepackage{array, longtable, pdflscape}
\newcolumntype{R}[1]{>{\raggedright\let\newline\\\arraybackslash\hspace{0pt}}m{#1}}

\begin{document}

\frenchspacing
\title{Planning, versie 0.1\\bt-planning-0.1-mh}
\author{Menno Hellinga}

\maketitle

Voorlopige planning tot 11 Dec. 2014. Ik kan pas een volledige planning maken
als we goed in beeld hebben wat er in de app komt en als de codeer-rollen
verdeeld zijn.

\begin{landscape}

\begin{table}
\small
\raggedright
\begin{longtable}{|R{0.15\textwidth} | R{0.2\textwidth} | R{0.2\textwidth} | R{0.2\textwidth} | R{0.2\textwidth} | R{0.2\textwidth} | R{0.2\textwidth} | R{0.2\textwidth}|}

\hline
\textbf{datum}	&\textbf{Menno}	&\textbf{Nander}	&\textbf{Dylan}	&\textbf{Jan Chr.}	&\textbf{Rik}	&\textbf{iedereen} &\textbf{Nota Bene}	\\
\hline
27 Nov. '14	&		& presentatie versie 0.1 af; contact met LGC	& werkomgeving kiezen; GitHub opzetten; reader af	& formaat- en broninformatie; DFD; contextdiag. & UI-design versie 0.1 af & & \\
\hline
4 Dec. '14		&				& & werkomgeving beschikbaar voor iedereen & ontwerp af & UI-design versie 1.0 af &	& Menno en Dylan afwezig \\
\hline
11 Dec. '14	&	volledige planning af; codeer-rollen verdelen (tijdens vergadering)	& PJ-IT contacteren; logo af	&	&	&	& Java/Android leren	&	\\
\hline

\end{longtable}
\end{table}

\end{landscape}

\end{document}
