\documentclass[a4paper, 10pt]{article}

\usepackage[dutch]{babel}
\usepackage{url}

\begin{document}

\frenchspacing
\title{Notulen 13 November 2014\\bt-not-141113-mh}
\author{Menno Hellinga}
\date{\empty}

\maketitle
\newpage

\tableofcontents
\newpage

\section{Hoe heet het?}

Blind Turtle:

BLazing Internet Notification Deliverer: Truly and Utterly Radical Technology
with Lots of Easter eggs.

\section{Wat doet het?}

\begin{itemize}
	\item notificaties ophalen en tonen
	(\url{www3.pj.nl/infoschermgymnasium} ?)
	\item roosterwijzigingen ophalen en tonen (automatisch inloggen op SOM?)
	\item contactinformatie docenten (alleen de docenten die je hebt? best
	te doen als het rooster in de app staat)
	\item zeepkist voor het LGC
	\item fucking veel easter eggs
\end{itemize}

\section{Waar draait het op?}
Android en iOS (zowel iPhone als iPad).

\section{Hoe wordt het becijferd?}
Met een groepscijfer.

\section{Rolverdeling}

\subsection{Projectleider}
Menno Hellinga is projectleider en moet:
\begin{itemize}
	\item notulen bijhouden van de vergaderingen,
	\item de vergaderingsagenda's vaststellen,
	\item de vergaderingen voorzitten,
	\item met een hamer zwaaien, en
	\item een planning maken.
\end{itemize}

\subsection{Accountmanager}
Nander Vilar Castellar is accountmanager en zal:
\begin{itemize}
	\item contact zoeken/onderhouden met PJ en opperhoofd IT,
	\item een leipe presentatie maken,
	\item zorgen voor Play Store- en App Store-accounts, en
	\item een logo tekenen/animeren.
\end{itemize}

\subsection{Technisch Analyst}
Dylan Rakiman moet als Tech. Anal.:
\begin{itemize}
	\item een werkomgeving kiezen (alsjeblieft geen Eclipse),
	\item een GitHub opzetten, en
	\item er voor zorgen dat iedereen toegang heeft tot werkomgeving en
		GitHub.
\end{itemize}

\subsection{Informatie-Analyst}
Jan Chr. Zwier gaat als Inf. Anal.:
\begin{itemize}
	\item contact opnemen met Karin ivm gegevens,
	\item op papier het volgende noteren:
	\begin{itemize}
		\item broninformatie
		\item formaatinformatie
	\end{itemize}
	\item rooster op scherm, en
	\item mededelingen.
\end{itemize}
geen idee wat Vrijst. met de laatste twee bedoelt

\subsection{Designer/Ontwerper}
Rik Wolters zal in deze functie:
\begin{itemize}
	\item een schets van de UI maken voor telefoon en tablet, in portrait en
		landscape, en
	\item een concept doen. (Vrijstijler, licht dit alsjeblieft toe.)
\end{itemize}

\section{Volgende vergadering}
De volgende vergadering vindt op donderdag 20 november 2014 plaats. De agenda
(\url{bt-agd-141120-0.1-mh.pdf}) zal uiterlijk woensdag 19 november rond worden
gemaild.

Vervul voor die tijd je taken to the fullest extent possible. Dit gaat geen
herhaling van de Romereis-app worden.

\end{document}
